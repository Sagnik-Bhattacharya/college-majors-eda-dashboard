\documentclass[11pt,a4paper]{article}

\usepackage{graphicx}
\usepackage{float}
\usepackage{booktabs}
\usepackage{geometry}
\usepackage{hyperref}
\geometry{margin=1in}

\title{\textbf{Economic Outcomes of U.S. College Majors}}
\author{Sagnik Bhattacharya}
\date{\today}

\begin{document}

\maketitle

\section*{Executive Summary}
This project presents an end-to-end data analysis of the economic outcomes
associated with U.S. college majors. Using Python for data cleaning and feature
engineering, SQL for analytical querying, and Excel for interactive dashboard
visualization, the analysis explores salary distributions, unemployment risk,
graduate education premiums, and gender representation across academic fields.
The goal is to provide data-driven insights that support informed educational
and career decision-making.

\section*{Dataset Overview}
The analysis integrates multiple datasets covering:
\begin{itemize}
    \item Employment and salary outcomes across all age groups
    \item Employment statistics for recent graduates
    \item Graduate versus non-graduate salary comparisons
    \item Gender representation across majors, including STEM fields
\end{itemize}

All datasets were cleaned, standardized, and merged using Python before being
loaded into a relational SQL database and an Excel-based analytical dashboard.

\section*{Data Source and Attribution}
The datasets used in this project were sourced from Kaggle and are derived from
the American Community Survey (ACS) conducted by the U.S. Census Bureau.

\begin{itemize}
    \item Dataset Title: \textit{Uncovering Insights to College Majors and Their Outcomes}
    \item Curated By: \textbf{The Devastator} (Kaggle)
    \item Original Source: U.S. Census Bureau — American Community Survey
    \item Platform: Kaggle
    \item URL: \url{https://www.kaggle.com/datasets/thedevastator/uncovering-insights-to-college-majors-and-their}
\end{itemize}

The data was used strictly for educational and analytical purposes. All credit
for data collection and original compilation belongs to the U.S. Census Bureau
and the dataset curator.

\section*{Dashboard Overview}
Figure~\ref{fig:dashboard} presents the interactive Excel dashboard summarizing
key metrics such as median salary, unemployment rate, graduate salary premium,
and gender representation. Slicers allow dynamic filtering by major category,
salary thresholds, and female participation.

\begin{figure}[H]
    \centering
    \includegraphics[width=\textwidth]{../assets/dashboard_overview.png}
    \caption{Excel Dashboard Overview: Economic Outcomes of College Majors}
    \label{fig:dashboard}
\end{figure}

\section*{Salary Analysis}
Median salary varies significantly across academic disciplines. Engineering and
technology-focused majors consistently rank among the highest-paying fields,
while arts and social science majors tend to report lower median earnings.

\begin{figure}[H]
    \centering
    \includegraphics[width=0.9\textwidth]{../assets/salary_analysis.png}
    \caption{Top Majors by Median Salary}
    \label{fig:salary_analysis}
\end{figure}

\begin{figure}[H]
    \centering
    \includegraphics[width=0.85\textwidth]{../assets/salary_by_category.png}
    \caption{Average Median Salary by Major Category}
    \label{fig:salary_category}
\end{figure}

\section*{Unemployment and Employment Risk}
High earning potential does not always guarantee employment stability. Certain
high-paying majors also exhibit elevated unemployment rates, indicating increased
career risk.

\begin{figure}[H]
    \centering
    \includegraphics[width=0.85\textwidth]{../assets/unemployment_by_category.png}
    \caption{Average Unemployment Rate by Major Category}
    \label{fig:unemployment_category}
\end{figure}

\begin{figure}[H]
    \centering
    \includegraphics[width=0.85\textwidth]{../assets/salary_by_unemployment.png}
    \caption{Median Salary vs Unemployment Rate by Major}
    \label{fig:salary_unemployment}
\end{figure}

\section*{Graduate Education Premium}
Graduate education generally provides a measurable earnings advantage. Technical
and professional majors demonstrate the highest graduate salary premiums,
reinforcing the economic value of advanced education in specialized fields.

\begin{figure}[H]
    \centering
    \includegraphics[width=0.85\textwidth]{../assets/salary_premium_by_category.png}
    \caption{Average Graduate Salary Premium by Major Category}
    \label{fig:salary_premium}
\end{figure}

\begin{figure}[H]
    \centering
    \includegraphics[width=0.85\textwidth]{../assets/grad_by_non_grad_salary.png}
    \caption{Graduate vs Non-Graduate Median Salary Comparison}
    \label{fig:grad_vs_nongrad}
\end{figure}

\section*{Gender Representation and Salary Trends}
Gender distribution across majors remains uneven. Majors with higher female
representation tend to exhibit lower median salaries, highlighting persistent
structural and occupational disparities within the labor market.

\begin{figure}[H]
    \centering
    \includegraphics[width=0.85\textwidth]{../assets/female_share_by_salary.png}
    \caption{Female Share vs Median Salary by Major}
    \label{fig:female_salary}
\end{figure}

\section*{Key Insights}
\begin{itemize}
    \item Engineering and STEM majors offer the highest median salaries overall.
    \item High salary does not necessarily correlate with low unemployment risk.
    \item Graduate degrees significantly enhance earning potential across most fields.
    \item Majors with higher female participation tend to have lower median salaries.
\end{itemize}

\section*{Conclusion}
This project demonstrates a complete analytical workflow combining Python, SQL,
and Excel to extract meaningful insights from real-world data. The resulting
dashboard and findings provide a practical decision-support tool for students,
educators, and policymakers evaluating academic and career pathways.

\end{document}
